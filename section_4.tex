\section{Proof Generation and Back-ends}

\subsection{Choosing a Back-end}

A back-end is automatically chosen based on the extensions of the generated files. If we want to generate a Halo2 circuit, we can simply ask it to make a \lstinline{.halo2} file. Repeating the examples from the introduction, we could have done the following to get a Halo2 circuit.

\begin{lstlisting}[language=bash]
  $ target/debug/vamp-ir compile -u examples/params.pp \
                                 -s examples/ex1.pir \
                                 -o examples/circuit.halo2
                                 
  > [...]
  > * Constraint compilation success!

  $ target/debug/vamp-ir prove -u examples/params.pp \
                               -c examples/circuit.halo2 \
                               -o examples/proof.halo2

  > [...]
  > * Proof generation success!

  $ target/debug/vamp-ir verify -u examples/params.pp \
                                -c examples/circuit.halo2 \
                                -p examples/proof.halo2

  > [...]
  > * Zero-knowledge proof is valid
\end{lstlisting}

Currently, \vampir\ supports the following proof systems;

\begin{itemize}
\item PLONK via \lstinline{.plonk}
\item Halo2 via \lstinline{.halo2}
\item \textcolor{red}{[TODO]}
\end{itemize}

\textcolor{red}{[TODO: What particularities are there with the different proof systems?]}

\textcolor{red}{[TODO: Contoling proofs, e.g. with the -m command.]}
